\documentclass[article]{uom-coursework}

\usepackage{blindtext}
\usepackage[showframe]{geometry}

\counterwithout{section}{chapter}

% Biblography
% \addbibresource{dsa.bib}

\title{Data Structures and\\Algorithms 2}
\tagline{Coursework}
\author{Juan Scerri}
\authorid{1234567A}
\courseworkname{Some Degree}
\doctype{coursework}
\courseworkdate{\monthyeardate\today}
\subjectcode{ICS2210}


\begin{document}

%----------------------------------
%	Front Matter
%----------------------------------

\pagestyle{umpage}

\frontmatter

\maketitle % Print the title page

\tableofcontents % Print the table of contents

\clearpage

\lstlistoflistings

\clearpage

\mainmatter

% \chapter{Report}

\chapter*{Report}
\label{chap:report}
\addcontentsline{toc}{chapter}{\nameref{chap:report}}

% TODO: the report
% add reason for language
% add comments for explanation
% add lstlisting describing the knuth_shuffle
% add lstlisting describing the untyped_tree
% add lstlisting describing the untyped_avl_tree
% add lstlisting describing the untyped_rb_tree
% add lstlisting describing the skip_list
% then explain the procedure.py and what is does
% add a section describing the statistics

\section{Language}

The programming language used for this coursework is Python 3.
The main reason for this choice is the speed of development.

\section{Comments}

Throughout the code an effort was made to deliver explanations
through comments. In Python comments start with a \#. Moreover,
in the listings comments have a grey colour.

\section{Knuth Shuffling}

\lstinputlisting[caption={A function implementing Knuth shuffling.},language=Python,firstline=8,lastline=13]{../ics2210/knuth_shuffling.py}

Knuth shuffling, also known as Fisher--Yates shuffling, is an
algorithm used for generating permutations of the elements of a
given array.

The implementation follows the pseudocode described here:
\url{https://en.wikipedia.org/wiki/Fisher%E2%80%93Yates_shuffle#The_modern_algorithm}

\section{Binary Search Tree}


\section{Section?}
\section{Section?}
\section{Section?}
\section{Section?}
\section{Section?}
\section{Section?}
\section{Section?}
\section{Section?}


\blindtext
\blindtext
\blindtext

\end{document}
